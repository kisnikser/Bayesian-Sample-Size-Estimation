\chapter{Постановка задачи}\label{chap1}

Задана выборка размера $m$:
\[ \mathfrak{D}_m = \left\{ \mathbf{x}_i, y_i \right\}_{i = 1}^{m}, \]
где $\mathbf{x}_i \in \mathbb{X}, y_i \in \mathbb{Y}$.
%В случае обобщенной линейной модели $\mathbb{X} = \mathbb{R}^n$. Множество $\mathbb{Y} = \mathbb{R}$ в задаче регрессии и $\mathbb{Y} = \left\{ 1, \ldots, K \right\}$ в задаче $K$-классовой классификации.

Введем параметрическое семейство $p(y | \mathbf{x}, \mathbf{w})$ для аппроксимации неизвестного апостериорного распределения $p(y | \mathbf{x})$ целевой переменной $y$ при известных признаковом описании объекта $\mathbf{x}$ и параметрах $\mathbf{w} \in \mathbb{W}$. 
%Для обобщенной линейной модели $\mathbb{W} = \mathbb{R}^n$.

Определим функцию правдоподобия и логарифмическую функцию правдоподобия выборки $\mathfrak{D}_m$:
\[ L(\mathfrak{D}_m, \mathbf{w}) = \prod_{i=1}^{m} p(y_i | \mathbf{x}_i, \mathbf{w}), \qquad l(\mathfrak{D}_m, \mathbf{w}) = \sum\limits_{i=1}^{m} \log p(y_i | \mathbf{x}_i, \mathbf{w}). \]

Оценим параметры, используя метод максимума правдоподобия:
\[ \hat{\mathbf{w}}_{m} = \arg\max_{\mathbf{w}} L(\mathfrak{D}_m, \mathbf{w}). \]

Требуется найти достаточный размер выборки $m^*$. При этом понятие достаточности может определяться различными способами. Часто оно дается в терминах функции правдоподобия и полученной из ее максимизации оценки параметров. Также стоит учесть, что возможно $m^* \leqslant m$ или $m^* > m$. Эти два случая будут отдельно рассмотрены далее.