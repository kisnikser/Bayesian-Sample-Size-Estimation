\begin{abstract}
	Исследуется задача выбора достаточного размера выборки. Рассматривается проблема определения достаточного размера выборки без постановки статистической гипотезы о распределении параметров модели. Предлагаются два подхода к определению достаточного размера выборки по значениям функции правдоподобия на подвыборках с возвращением. Эти подходы основываются на эвристиках о поведении функции правдоподобия при большом количестве объектов в выборке. Предлагаются два подхода к определению достаточного размера выборки на основании близости апостериорных распределений параметров модели на схожих подвыборках. Доказывается корректность представленных подходов при определенных ограничениях на используемую модель. Доказывается теорема о моментах предельного апостериорного распределения параметров в модели линейной регрессии. Предлагается метод прогнозирования функции правдоподобия в случае недостаточного размера выборки. Проводится вычислительный эксперимент для анализа свойств предложенных методов.
\end{abstract}

\keywords{достаточный размер выборки \and байесовский вывод \and бутстрапирование \and функция близости апостериорных распределений \and линейная регрессия}