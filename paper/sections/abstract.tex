\begin{abstract}
	Исследуется задача выбора достаточного размера выборки. Рассматривается проблема определения достаточного размера выборки без учета природы параметров используемой модели. Предлагается использовать функцию правдоподобия выборки. Используются подходы на основе эвристик о поведении функции правдоподобия при достаточном количестве объектов в выборке. Проводится вычислительный эксперимент для анализа свойств предложенных методов.
\end{abstract}

\keywords{определение размера выборки \and байесовский подход}