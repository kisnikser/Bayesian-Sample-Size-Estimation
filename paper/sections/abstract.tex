\begin{abstract}
	Исследуется задача выбора достаточного размера выборки. Рассматривается проблема определения достаточного размера выборки без учета природы параметров используемой модели. Предлагается использовать функцию правдоподобия выборки. Рассматриваются два подхода к определению достаточного размера выборки через функцию правдоподобия. Предлагаемые подходы основываются на эвристиках о поведении функции правдоподобия при большом количестве объектов в выборке. Доказывается корректность предложенных подходов при определенных ограничениях на используемую модель. Предлагается метод прогнозирования функции правподобия в случае недостаточного размера выборки. Проводится вычислительный эксперимент для анализа свойств предложенных методов.
\end{abstract}

\keywords{определение размера выборки \and байесовский подход}