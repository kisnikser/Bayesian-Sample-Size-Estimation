\section{Введение}

Задача машинного обучения с учителем предполагает выбор предсказательной модели из некоторого параметрического семейства. Обычно такой выбор связан с некоторыми статистическими гипотезами, например, максимизацией некоторого функционала качества. 
\begin{definition}
    Модель прогнозирования, которая соответствует этим статистическим гипотезам, называется \textbf{адекватной} моделью.
\end{definition}

При проведении эксперимента зачастую дана конечная обучающая выборка.

\begin{definition}
    Размер выборки, необходимый для построения адекватной модели прогнозирования, называется \textbf{достаточным}.
\end{definition}

В работе \cite{Grabovoy2022} представлены десять методов для оценки достаточного размера выборки. Среди них есть как статистические, так и байесовские подходы. 