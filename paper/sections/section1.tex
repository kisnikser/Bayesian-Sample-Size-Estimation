\section{Постановка задачи}\label{sec1}

Объектом называется пара $(\bx, y)$, где $\bx \in \mathbb{X} \subseteq \mathbb{R}^n$ есть вектор признакового описания объекта, а $y \in \mathbb{Y}$ есть значение целевой переменной. В задаче регрессии $\mathbb{Y} = \mathbb{R}$, а в задаче $K$-классовой классификации $\mathbb{Y} = \{1, \ldots, K\}$.

Матрицей объекты-признаки для выборки $\mathfrak{D}_m = \left\{ (\bx_i, y_i) \right\}, i \in \mathcal{I} = \{ 1, \ldots, m \}$ размера $m$ называется матрица $\bX = \left[ \bx_1, \ldots, \bx_m \right]\T \in \mathbb{R}^{m \times n}$.

Вектором ответов (вектором значений целевой переменной) для выборки $\mathfrak{D}_m = \left\{ (\bx_i, y_i) \right\}, i \in \mathcal{I} = \{ 1, \ldots, m \}$ размера $m$ называется вектор $\by = \left[ y_1, \ldots, y_m \right]\T \in \mathbb{Y}^m$.

Моделью называется параметрическое семейство функций $f$, отображающих декартово произведение множества значений признакового описания объектов $\mathbb{X}$ и множества значений параметров $\mathbb{W}$ во множество значений целевой переменной $\mathbb{Y}$: $$f: \mathbb{X} \times \mathbb{W} \to \mathbb{Y}.$$

Вероятностной моделью называется совместное распределение вида $$p(y, \bw | \bx) = p(y | \bx, \bw) p(\bw): \mathbb{Y} \times \mathbb{W} \times \mathbb{X} \to \mathbb{R}^+,$$
где $\bw \in \mathbb{W}$ есть набор параметров модели, $p(y | \bx, \bw)$ задает правдоподобие объекта, а $p(\bw)$ задает априорное распределение параметров.

Функцией правдоподобия простой выборки $\mathfrak{D}_m = \left\{ (\bx_i, y_i) \right\}, i \in \mathcal{I} = \{ 1, \ldots, m \}$ размера $m$ называется функция $$L(\mathfrak{D}_m, \mathbf{w}) = p(\by | \bX, \bw) = \prod_{i=1}^{m} p(y_i | \mathbf{x}_i, \mathbf{w}).$$ Ее логарифм $$l(\mathfrak{D}_m, \mathbf{w}) = \sum\limits_{i=1}^{m} \log p(y_i | \mathbf{x}_i, \mathbf{w})$$ называется логарифмической функцией правдоподобия. Далее, если не оговорено противное, будем считать выборку простой.

Оценкой максимума правдоподобия набора параметров $\bw \in \mathbb{W}$ по выборке $\mathfrak{D}_m = \left\{ (\bx_i, y_i) \right\}, i \in \mathcal{I} = \{ 1, \ldots, m \}$ размера $m$ называется $$\hat{\mathbf{w}}_{m} = \argmax_{\bw \in \mathbb{W}} L(\mathfrak{D}_m, \mathbf{w}).$$

Ставится задача определения достаточного размера выборки $m^*$. Пусть задан некоторый критерий $T$. Он может быть построен, например, на основе эвристик о поведении параметров модели.
\begin{definition}
    Размер выборки $m^*$ называется \textbf{достаточным} согласно критерию $T$, если $T$ выполняется для всех $k \geqslant m^*$.
\end{definition}
Стоит учесть, что возможно $m^* \leqslant m$ или $m^* > m$. Эти два случая будут отдельно рассмотрены далее.