\section{Достаточный размер выборки не превосходит доступный}\label{sec2}

В этой главе будем считать, что достоверно $m^* \leqslant m$.

Рассмотрим выборку $\mathfrak{D}_k$ размера $k \leqslant m$. Оценим на ней параметры, используя метод максимума правдоподобия:
\[ \hat{\mathbf{w}}_{k} = \arg\max_{\mathbf{w}} L(\mathfrak{D}_k, \mathbf{w}). \]

Поскольку природа $\mathbf{w}$ нам неизвестна, для определения достаточности будем использовать функцию правдоподобия.

Когда в наличии имеется достаточно объектов, вполне естественно ожидать, что от одной реализации выборки к другой полученная оценка параметров не будет сильно меняться \cite{Joseph1997, Joseph1995}. То же можно сказать и про функцию правдоподобие. Таким образом, сформулируем, какой размер выборки можно считать достаточным.

\begin{definition}
    \label{sufficient-variance}
    Зафиксируем некоторое положительное число $\varepsilon > 0$. Размер выборки $m^*$ называется \textbf{D-достаточным}, если для любого $k \geqslant m^*$
    \[ D(k) = \mathbb{D}_{\mathfrak{D}_k} L(\mathfrak{D}_m, \hat{\mathbf{w}}_{k}) \leqslant \varepsilon. \]
\end{definition}
\begin{note}
    В определении~\ref{sufficient-variance} вместо функции правдоподобия $L(\mathfrak{D}_m, \hat{\mathbf{w}}_{k})$ можно рассматривать ее логарифм $l(\mathfrak{D}_m, \hat{\mathbf{w}}_{k})$.
\end{note}

С другой стороны, когда в наличии имеется достаточно объектов, также вполне естественно, что при добавлении очередного объекта в рассмотрение полученная оценка параметров не будет сильно меняться. Сформулируем еще одно определение.

\begin{definition}
    \label{sufficient-difference}
    Зафиксируем некоторое положительное число $\varepsilon > 0$. Размер выборки $m^*$ называется \textbf{M-достаточным}, если для любого $k \geqslant m^*$ 
    \[ M(k) = \left| \mathbb{E}_{\mathfrak{D}_{k+1}} L(\mathfrak{D}_m, \hat{\mathbf{w}}_{k+1}) - \mathbb{E}_{\mathfrak{D}_k} L(\mathfrak{D}_m, \hat{\mathbf{w}}_{k}) \right| \leqslant \varepsilon. \]
\end{definition}
\begin{note}
    В определении~\ref{sufficient-difference} вместо функции правдоподобия $L(\mathfrak{D}_m, \hat{\mathbf{w}}_{k})$ можно рассматривать ее логарифм $l(\mathfrak{D}_m, \hat{\mathbf{w}}_{k})$.
\end{note}

\textcolor{red}{Как доказать корректность этих определений? А именно, почему такой размер выборки существует?}

По условию задана одна выборка. Поэтому в эксперименте нет возможности посчитать указанные в определениях математическое ожидание и дисперсию. Для их оценки воспользуемся техникой бутстрэп. А именно, сгенерируем из заданной $\mathfrak{D}_m$ некоторое число $B$ подвыборок размера $k$ с возвращением. Для каждой из них получим оценку параметров $\hat{\mathbf{w}}_{k}$ и посчитаем значение $L(\mathfrak{D}_m, \hat{\mathbf{w}}_{k})$. Для оценки будем использовать выборочное среднее и несмещенную выборочную дисперсию (по бутстрэп-выборкам). \textcolor{red}{Как доказать <<хорошие>> свойства этих оценок?}