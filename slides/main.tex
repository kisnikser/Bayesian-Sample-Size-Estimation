\documentclass{beamer}

\usepackage{arxiv}

\usepackage[T2A]{fontenc}
\usepackage[utf8]{inputenc}
\usepackage[english, russian]{babel}
% \usepackage{cmap}
\usepackage{url}
\usepackage{booktabs}
\usepackage{nicefrac}
\usepackage{microtype}
\usepackage{lipsum}
\usepackage{graphicx}
\usepackage{subfig}
\usepackage[square,sort,comma,numbers]{natbib}
\usepackage{doi}
\usepackage{multicol}
\usepackage{multirow}
\usepackage{tabularx}

\usepackage{tikz}
\usetikzlibrary{matrix}

% Algorithms
\usepackage{algpseudocode}
\usepackage{algorithm}

%% Шрифты
\usepackage{euscript} % Шрифт Евклид
\usepackage{mathrsfs} % Красивый матшрифт
\usepackage{extsizes}

\usepackage{makecell} % diaghead in a table
\usepackage{amsmath,amsfonts,amssymb,amsthm,mathtools,dsfont}
\usepackage{icomma}

\usepackage{hyperref}
% \usepackage[usenames,dvipsnames,svgnames,table,rgb]{xcolor}

\hypersetup{
	unicode=true,
	pdftitle={Байесовский подход к выбору достаточного размера выборки},
	pdfauthor={Киселев Никита Сергеевич},
	pdfkeywords={определение размера выборки, байесовский подход},
	colorlinks=true,
	linkcolor=black,        % внутренние ссылки
	citecolor=green,         % на библиографию
	filecolor=magenta,      % на файлы
	urlcolor=blue           % на URL
}

\graphicspath{{./figures/}}

\usepackage{enumitem} % Для модификаций перечневых окружений
\usepackage{etoolbox}

\makeatletter
\expandafter\patchcmd\csname\string\algorithmic\endcsname{\itemsep\z@}{\itemsep=1.5mm}{}{}
\makeatother

% Теоремы
\newtheorem{theorem}{Теорема}
\newtheorem{lemma}{Лемма}
\newtheorem{proposition}{Утверждение}
\newtheorem*{exercise}{Упражнение}
\newtheorem*{problem}{Задача}

\newtheorem{definition}{Определение}
\newtheorem*{corollary}{Следствие}
\newtheorem*{note}{Замечание}
\newtheorem*{reminder}{Напоминание}
\newtheorem*{example}{Пример}
\newtheorem*{cexample}{Контрпример}
\newtheorem*{solution}{Решение}

\renewcommand{\abstractname}{Аннотация}
\createthesistitle

\begin{document}

%=======
\begin{frame}[noframenumbering,plain]
	%\thispagestyle{empty}
	\titlepage
\end{frame}
%=======
\begin{frame}{Байесовский выбор достаточного размера выборки}
    Исследуется задача выбора достаточного размера выборки.
    \vfill
    \begin{block}{Проблема}
        Большинство подходов используют распределение параметров модели. Статистические методы требуют для оценки избыточный размер доступной выборки.
    \end{block}
    \vfill
    \begin{block}{Цель}
        Требуется предложить метод, не использующий напрямую параметры модели. Необходимо учесть недостаточный размер доступной выборки.
    \end{block}
    \vfill
    \begin{block}{Решение}
        Предлагается использовать функцию правдоподобия выборки. Рассматривается подход в случаях избыточного и недостаточного размеров доступной выборки. 
    \end{block}
\end{frame}
%=======
\begin{frame}{Постановка задачи выбора размера выборки}
    \begin{block}{Выборка}
        \vspace{-0.3cm}
        \[ \mathfrak{D}_m = \left\{ \bx_i, y_i \right\}_{i = 1}^{m}, \ \bx_i \in \mathbb{X}, \ y_i \in \mathbb{Y}. \]
        \vspace{-0.7cm}
    \end{block}
    \begin{block}{Параметризация распределения}
        \vspace{-0.3cm}
        \[ p(y | \bx) \quad \longrightarrow \quad p(y | \bx, \bw), \ \bw \in \mathbb{W}. \]
        \vspace{-0.7cm}
    \end{block}
    \begin{block}{Функция правдоподобия выборки}
        \vspace{-0.5cm}
        \[ L(\mathfrak{D}_m, \bw) = \prod_{i=1}^{m} p(y_i | \bx_i, \bw), \qquad l(\mathfrak{D}_m, \bw) = \sum\limits_{i=1}^{m} \log p(y_i | \bx_i, \bw). \]
        \vspace{-0.5cm}
    \end{block}
    \begin{block}{Оценка максимального правдоподобия}
        \vspace{-0.3cm}
        \[ \hat{\bw}_{m} = \arg\max_{\bw} L(\mathfrak{D}_m, \bw). \]
        \vspace{-0.7cm}
    \end{block}
    \begin{block}{Цель}
        Требуется определить достаточный размер выборки $m^*$.
    \end{block}
\end{frame}
%=======
\begin{frame}{Достаточный размер выборки не превосходит доступный}
    Рассмотрим выборку $\mathfrak{D}_k$ размера $k \leqslant m$. Оценим на ней параметры, используя метод максимума правдоподобия:
    \[ \hat{\mathbf{w}}_{k} = \arg\max_{\mathbf{w}} L(\mathfrak{D}_k, \mathbf{w}). \]
    Зафиксируем некоторое положительное число $\varepsilon > 0$.
    \begin{rusdefinition}[D-достаточный размер выборки]
        Размер выборки $m^*$ называется \textbf{D-достаточным}, если для любого $k \geqslant m^*$
        \[ D(k) = \mathbb{D}_{\mathfrak{D}_k} L(\mathfrak{D}_m, \hat{\mathbf{w}}_{k}) \leqslant \varepsilon. \]
    \end{rusdefinition}
    \begin{rusdefinition}[M-достаточный размер выборки]
        Размер выборки $m^*$ называется \textbf{M-достаточным}, если для любого $k \geqslant m^*$ 
        \[ M(k) = \left| \mathbb{E}_{\mathfrak{D}_{k+1}} L(\mathfrak{D}_m, \hat{\mathbf{w}}_{k+1}) - \mathbb{E}_{\mathfrak{D}_k} L(\mathfrak{D}_m, \hat{\mathbf{w}}_{k}) \right| \leqslant \varepsilon. \]
    \end{rusdefinition}
\end{frame}
%=======
\begin{frame}{Достаточный размер выборки больше доступного}
    \begin{block}{}
        Возникает задача прогнозирования математического ожидания и функции правдоподобия при $k > m$.
    \end{block}
    \begin{figure}[h!]
        \centering
        \includegraphics[width=\textwidth]{paper/figures/image.pdf}
        \label{image}
    \end{figure}
\end{frame}
%=======
\begin{frame}{Синтетическая выборка при $m^* \leqslant m$}
    \begin{center}
        Линейная регрессия
    \end{center}
    \begin{figure}[h!]
        \centering
        \includegraphics[width=\textwidth]{paper/figures/synthetic-regression-sufficient.pdf}
        \label{synthetic-regression-sufficient}
    \end{figure}
    \vspace{-1cm}
    \begin{center}
        Логистическая регрессия
    \end{center}
    \begin{figure}[h!]
        \centering
        \includegraphics[width=\textwidth]{paper/figures/synthetic-classification-sufficient.pdf}
        \label{synthetic-classification-sufficient}
    \end{figure}
\end{frame}
%=======
\begin{frame}{Синтетическая выборка при $m^* > m$}
    Для синтетических выборок проведена аппроксимация функций правдоподобия. Среднее значение и дисперсия аппроксимированы соответственно функциями
    \[ \varphi(m) = a_1 - a_2^2 \exp\left( - a_3^2 m \right) - \dfrac{a_4^2}{m^{3/2}} \]
    и
    \[ \psi(m) = b_1^2 \exp\left( - b_2^2 m \right) + \dfrac{b_3^2}{m^{3/2}}, \]
    где $\mathbf{a}$ и $\mathbf{b}$~--- вектора параметров.
\end{frame}
%=======
\begin{frame}{Синтетическая выборка при $m^* > m$}
    \begin{center}
        Линейная регрессия
    \end{center}
    \begin{figure}[h!]
        \centering
        \includegraphics[width=\textwidth]{paper/figures/synthetic-regression-approximation.pdf}
        \label{synthetic-regression-approximation}
    \end{figure}
    \vspace{-1cm}
    \begin{center}
        Логистическая регрессия
    \end{center}
    \begin{figure}[h!]
        \centering
        \includegraphics[width=\textwidth]{paper/figures/synthetic-classification-approximation.pdf}
        \label{synthetic-classification-approximation}
    \end{figure}
\end{frame}
%=======
\begin{frame}{Дальнейшие цели}
    \begin{itemize}
        \item Доказать корректность предложенных определений.
        \item Доказать <<хорошие>> свойства бутстрэп-оценок математического ожидания и дисперсии функции правдоподобия выборки.
        \item Улучшить подход к прогнозированию функции правдоподобия при $m^* > m$.
    \end{itemize}
\end{frame}
%=======
%\begin{frame}{Выносится на защиту}
%    \begin{enumerate}
%        \item Это первое...
%        \vfill
%        \item Это второе...
%        \vfill
%        \item Это третье...
%    \end{enumerate}
%\end{frame}
%=======

\end{document} 